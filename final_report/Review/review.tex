\documentclass[finalReport.tex]{subfiles}
\begin{document}
\chapter{Review}\label{ch:review}

Communication has evolved. Chat programs prior to this evolution were restricted to dainty, dull and dire chat rooms. The surge of modern mobile technology (and this technology becoming widely available for GEN Y), the introduction and high demand of apps, and improvements in data structures and bandwidth has given birth to a vast array of mobile and desktop chat applications. There currently is an oversaturation of chat programs that are widely available. In this section of the report we discuss a selection of the current batch of distributed chat programs available, and what makes these forms of communication so appealing, accessible, convenient and exciting.

A distributed chat system involves the use of clients and a server (or a cluster of servers) to build a form of communication between those clients. Modern day technology allows this concept to be expanded; making chat programs the primary forms of communication. Below are examples of chat programs from which the group drew inspiration from to build \textit{Synomilia chat}.

\section{WhatsApp}
The most popular chat program \textit{WhatsApp} acted as the bridge between the general public (who previously used SMS text messaging as the medium for mobile communication) and distributed chat systems. The simplicity is what made \textit{WhatsApp} accessible, and eventually what made it the most popular around the world.
 
A new user can simply install the application on their phone and use it instantly. The main interface consists of a list of chats the user has open and another list of all the contacts who have \textit{WhatsApp} installed on their phone. Once the user clicks on a contact/existing chat, they are taken to the chat screen. The chat screen, like most chat screens is instantly recognisable. The user types in a message and when they send this, this message is displayed on the left of the screen, and when they receive an incoming message, this is displayed on the right.
The group drew inspiration from \textit{WhatsApp}’s simplicity. One of the main objectives of the team was to make \textit{Synomilia chat} accessible, and this was done by following \textit{WhatsApp}’s interface as a blueprint for the Android application.

\section{Google Talk and Skype}
The group discussed ideas about the authentication and logging in to the system. This was an essential part of \textit{Synomilia chat}. \textit{WhatsApp} uses a user's mobile phone number to login to the system, as the team also developed a desktop client; this login procedure was not feasible. Therefore, the team used a username authentication method which is used by \textit{Skype} and \textit{Google Talk}.
\textit{Google Talk}'s interface is simple to use, with minimalist designs and a lack of clutter making it a popular application among desktop users. This was crucial, as the team wanted to make an application which was easy to use on both Android and the desktop application, following a similar design structure on both clients was essential. This allowed the team to use the \textit{Google Talk} interface as a design structure for the desktop client.





\end{document}